%%%%%%%%%%%%%%%%%%%%%%%%%%%%%%%%%%%%%%%%%%%%%%%%%%%%%%%%%%%%%%%%%%%%%%%%
%%%%%%%%%%%%%%%%%%%%%% Simple LaTeX CV Template %%%%%%%%%%%%%%%%%%%%%%%%
%%%%%%%%%%%%%%%%%%%%%%%%%%%%%%%%%%%%%%%%%%%%%%%%%%%%%%%%%%%%%%%%%%%%%%%%

%%%%%%%%%%%%%%%%%%%%%%%%%%%%%%%%%%%%%%%%%%%%%%%%%%%%%%%%%%%%%%%%%%%%%%%%
%% NOTE: If you find that it says                                     %%
%%                                                                    %%
%%                           1 of ??                                  %%
%%                                                                    %%
%% at the bottom of your first page, this means that the AUX file     %%
%% was not available when you ran LaTeX on this source. Simply RERUN  %%
%% LaTeX to get the ``??'' replaced with the number of the last page  %%
%% of the document. The AUX file will be generated on the first run   %%
%% of LaTeX and used on the second run to fill in all of the          %%
%% references.                                                        %%
%%%%%%%%%%%%%%%%%%%%%%%%%%%%%%%%%%%%%%%%%%%%%%%%%%%%%%%%%%%%%%%%%%%%%%%%

%%%%%%%%%%%%%%%%%%%%%%%%%%%%%%%%%%%%%%%%%%%%%%%%%%%%%%%%%%%%%%%%%%%%%%%%
%% NOTE: If you are getting compilation errors referring to list      %%
%%       definitions that don't match, you may need to upgrade to a   %%
%%       newer version of the enumitem package. Try going to:         %%
%%                                                                    %%
%%   http://www.ctan.org/tex-archive/macros/latex/contrib/enumitem    %%
%%                                                                    %%
%%       then download the enumitem.sty file from there. Place it in  %%
%%       the same directory as your CV. So long as there are no other %%
%%       conflicts with older packages on your system, hopefully that %%
%%       will fix your compilation problems.                          %%
%%%%%%%%%%%%%%%%%%%%%%%%%%%%%%%%%%%%%%%%%%%%%%%%%%%%%%%%%%%%%%%%%%%%%%%%

%%%%%%%%%%%%%%%%%%%%%%%%%%%% Document Setup %%%%%%%%%%%%%%%%%%%%%%%%%%%%

% Don't like 10pt? Try 11pt or 12pt
\documentclass[10pt]{article}

% The automated optical recognition software used to digitize resume
% information works best with fonts that do not have serifs. This
% command uses a sans serif font throughout. Uncomment both lines (or at
% least the second) to restore a Roman font (i.e., a font with serifs).
%\usepackage{times}
%\renewcommand{\familydefault}{\sfdefault}

% This is a helpful package that puts math inside length specifications
\usepackage{calc}
\usepackage{xcolor}
\usepackage{multicol}

% Layout: Puts the section titles on left side of page
\reversemarginpar

%
%         PAPER SIZE, PAGE NUMBER, AND DOCUMENT LAYOUT NOTES:
%
% The next \usepackage line changes the layout for CV style section
% headings as marginal notes. It also sets up the paper size as either
% letter or A4. By default, letter was used. If A4 paper is desired,
% comment out the letterpaper lines and uncomment the a4paper lines.
%
% As you can see, the margin widths and section title widths can be
% easily adjusted.
%
% ALSO: Notice that the includefoot option can be commented OUT in order
% to put the PAGE NUMBER *IN* the bottom margin. This will make the
% effective text area larger.
%
% IF YOU WISH TO REMOVE THE ``of LASTPAGE'' next to each page number,
% see the note about the +LP and -LP lines below. Comment out the +LP
% and uncomment the -LP.
%
% IF YOU WISH TO REMOVE PAGE NUMBERS, be sure that the includefoot line
% is uncommented and ALSO uncomment the \pagestyle{empty} a few lines
% below.
%

%% Use these lines for letter-sized paper
%\usepackage[paper=letterpaper,
%            %includefoot, % Uncomment to put page number above margin
%            marginparwidth=1.1in,     % Length of section titles
%            marginparsep=.05in,       % Space between titles and text
%            margin=0.6in,               % 1 inch margins
%            includemp]{geometry}

%% Use these lines for A4-sized paper
\usepackage[paper=a4paper,
            %includefoot, % Uncomment to put page number above margin
            marginparwidth=28mm,    % Length of section titles
            marginparsep=1.5mm,       % Space between titles and text
            margin=12mm,              % 25mm margins
            includemp]{geometry}

%% More layout: Get rid of indenting throughout entire document
\setlength{\parindent}{0in}

\usepackage[shortlabels]{enumitem}

% Simpler bibsections for CV sections
% (thanks to natbib for inspiration)
%
% * For lists of references with hanging indents and no numbers:
%
%   \begin{bibsection}
%       \item ...
%   \end{bibsection}
%
% * For numbered lists of references (with hanging indents):
%
%   \begin{bibenum}
%       \item ...
%   \end{bibenum}
%
%   Note that bibenum numbers continuously throughout. To reset the
%   counter, use
%
%   \restartlist{bibenum}
%
%   at the place where you want the numbering to reset.

\makeatletter
\newlength{\bibhang}
\setlength{\bibhang}{1em}
\newlength{\bibsep}
 {\@listi \global\bibsep\itemsep \global\advance\bibsep by\parsep}
\newlist{bibsection}{itemize}{3}
\setlist[bibsection]{label=,leftmargin=\bibhang}
\newlist{bibenum}{enumerate}{3}
\setlist[bibenum]{resume,label=[\arabic*],leftmargin={\bibhang+\widthof{[999]}}}
\setlist*[bibsection,bibenum]{%
        itemindent=-\bibhang,
        itemsep=\bibsep,parsep=\z@,partopsep=0pt,
        topsep=0pt}
\let\oldendbibenum\endbibenum
\def\endbibenum{\oldendbibenum\vspace{-.6\baselineskip}}
\let\oldendbibsection\endbibsection
\def\endbibsection{\oldendbibsection\vspace{-.6\baselineskip}}
\makeatother

%% Reference the last page in the page number
%
% NOTE: comment the +LP line and uncomment the -LP line to have page
%       numbers without the ``of ##'' last page reference)
%
% NOTE: uncomment the \pagestyle{empty} line to get rid of all page
%       numbers (make sure includefoot is commented out above)
%
\usepackage{fancyhdr,lastpage}
\pagestyle{fancy}
\pagestyle{empty}      % Uncomment this to get rid of page numbers
\fancyhf{}\renewcommand{\headrulewidth}{0pt}
\fancyfootoffset{\marginparsep+\marginparwidth}
\newlength{\footpageshift}
\setlength{\footpageshift}
          {0.5\textwidth+0.5\marginparsep+0.5\marginparwidth-2in}
\lfoot{\hspace{\footpageshift}%
       \parbox{4in}{\, \hfill %
                    \arabic{page} of \protect\pageref*{LastPage} % +LP
%                    \arabic{page}                               % -LP
                    \hfill \,}}

% Finally, give us PDF bookmarks
\usepackage{color,hyperref}
\definecolor{darkblue}{rgb}{0.0,0.0,0.3}
\hypersetup{colorlinks,breaklinks,
            linkcolor=darkblue,urlcolor=darkblue,
            anchorcolor=darkblue,citecolor=darkblue}

%%%%%%%%%%%%%%%%%%%%%%%% End Document Setup %%%%%%%%%%%%%%%%%%%%%%%%%%%%


%%%%%%%%%%%%%%%%%%%%%%%%%%% Helper Commands %%%%%%%%%%%%%%%%%%%%%%%%%%%%

%%% HEADING AT TOP OF CURRICULUM VITAE

% The title (name) with a horizontal rule under it
% (optional argument typesets an object right-justified across from name
%  as well)
%
% Usage: \makeheading{name}
%        OR
%        \makeheading[right_object]{name}
%
% Place at top of document. It should be the first thing.
% If ``right_object'' is provided in the square-braced optional
% argument, it will be right justified on the same line as ``name'' at
% the top of the CV. For example:
%
%       \makeheading[\emph{Curriculum vitae}]{Your Name}
%
% will put an emphasized ``Curriculum vitae'' at the top of the document
% as a title. Likewise, a picture could be included:
%
%   \makeheading[\includegraphics[height=1.5in]{my_picutre}]{Your Name}
%
% the picture will be flush right across from the name.
\newcommand{\makeheading}[2][]%
        {\hspace*{-\marginparsep minus \marginparwidth}%
         \begin{minipage}[t]{\textwidth+\marginparwidth+\marginparsep}%
             {\Large \bfseries #2 \hfill #1}\\[-0.15\baselineskip]%
                 \rule{\columnwidth}{1pt}%
         \end{minipage}}

%%% SECTION HEADINGS

% The section headings. Flush left in small caps down pseudo-margin.
%
% Usage: \section{section name}
\renewcommand{\section}[1]{\pagebreak[3]%
    \hyphenpenalty=10000%
    \vspace{1.3\baselineskip}%
    \phantomsection\addcontentsline{toc}{section}{#1}%
    \noindent\llap{\scshape\smash{\parbox[t]{\marginparwidth}{\raggedright #1}}}%
    \vspace{-\baselineskip}\par}

%%% LISTS
% This macro alters a list by removing some of the space that follows the list
% (is used by lists below)
\newcommand*\fixendlist[1]{%
    \expandafter\let\csname preFixEndListend#1\expandafter\endcsname\csname end#1\endcsname
    \expandafter\def\csname end#1\endcsname{\csname preFixEndListend#1\endcsname\vspace{-0.6\baselineskip}}}

% These macros help ensure that items in outer-type lists do not get
% separated from the next line by a page break
% (they are used by lists below)
\let\originalItem\item
\newcommand*\fixouterlist[1]{%
    \expandafter\let\csname preFixOuterList#1\expandafter\endcsname\csname #1\endcsname
    \expandafter\def\csname #1\endcsname{\csname preFixOuterList#1\endcsname\let\oldItem\item\def\item{\pagebreak[2]\oldItem}}
    \expandafter\let\csname preFixOuterListend#1\expandafter\endcsname\csname end#1\endcsname
    \expandafter\def\csname end#1\endcsname{\let\item\oldItem\csname preFixOuterListend#1\endcsname}}
\newcommand*\fixinnerlist[1]{%
    \expandafter\let\csname preFixInnerList#1\expandafter\endcsname\csname #1\endcsname
    \expandafter\def\csname #1\endcsname{\let\oldItem\item\let\item\originalItem\csname preFixInnerList#1\endcsname}
    \expandafter\let\csname preFixInnerListend#1\expandafter\endcsname\csname end#1\endcsname
    \expandafter\def\csname end#1\endcsname{\csname preFixInnerListend#1\endcsname\let\item\oldItem}}

% An itemize-style list with lots of space between items
%
% Usage:
%   \begin{outerlist}
%       \item ...    % (or \item[] for no bullet)
%   \end{outerlist}
\newlist{outerlist}{itemize}{3}
    \setlist[outerlist]{label=\enskip\textbullet,leftmargin=*}
    \fixendlist{outerlist}
    \fixouterlist{outerlist}

% An environment IDENTICAL to outerlist that has better pre-list spacing
% when used as the first thing in a \section
%
% Usage:
%   \begin{lonelist}
%       \item ...    % (or \item[] for no bullet)
%   \end{lonelist}
\newlist{lonelist}{itemize}{3}
    \setlist[lonelist]{label=\enskip\textbullet,leftmargin=*,partopsep=0pt,topsep=0pt}
    \fixendlist{lonelist}
    \fixouterlist{lonelist}
    
    

% An itemize-style list with little space between items
%
% Usage:
%   \begin{innerlist}
%       \item ...    % (or \item[] for no bullet)
%   \end{innerlist}
\newlist{innerlist}{itemize}{3}
    \setlist[innerlist]{label=\enskip\textbullet,leftmargin=*,parsep=0pt,itemsep=0pt,topsep=0pt,partopsep=0pt}
    \fixinnerlist{innerlist}

% An environment IDENTICAL to innerlist that has better pre-list spacing
% when used as the first thing in a \section
%
% Usage:
%   \begin{loneinnerlist}
%       \item ...    % (or \item[] for no bullet)
%   \end{loneinnerlist}
\newlist{loneinnerlist}{itemize}{3}
    \setlist[loneinnerlist]{label=\enskip\textbullet,leftmargin=*,parsep=0pt,itemsep=0pt,topsep=0pt,partopsep=0pt}
    \fixendlist{loneinnerlist}
    \fixinnerlist{loneinnerlist}

%%% EXTRA SPACE

% To add some paragraph space between lines.
% This also tells LaTeX to preferably break a page on one of these gaps
% if there is a needed pagebreak nearby.
\newcommand{\blankline}{\quad\pagebreak[3]}
\newcommand{\halfblankline}{\quad\vspace{-0.5\baselineskip}\pagebreak[3]}

%%% FORMATTING MACROS

% Uses hyperref to link DOI
\newcommand\doilink[1]{\href{http://dx.doi.org/#1}{#1}}
\newcommand\doi[1]{doi:\doilink{#1}}

% For \url{SOME_URL}, links SOME_URL to the url SOME_URL
\providecommand*\url[1]{\href{#1}{#1}}
% Same as above, but pretty-prints SOME_URL in teletype fixed-width font
\renewcommand*\url[1]{\href{#1}{\texttt{#1}}}

% For \email{ADDRESS}, links ADDRESS to the url mailto:ADDRESS
\providecommand*\email[1]{\href{mailto:#1}{#1}}
% Same as above, but pretty-prints ADDRESS in teletype fixed-width font
%\renewcommand*\email[1]{\href{mailto:#1}{\texttt{#1}}}

%\providecommand\BibTeX{{\rm B\kern-.05em{\sc i\kern-.025em b}\kern-.08em
%    T\kern-.1667em\lower.7ex\hbox{E}\kern-.125emX}}
%\providecommand\BibTeX{{\rm B\kern-.05em{\sc i\kern-.025em b}\kern-.08em
%    \TeX}}
\providecommand\BibTeX{{B\kern-.05em{\sc i\kern-.025em b}\kern-.08em
    \TeX}}
\providecommand\Matlab{\textsc{Matlab}}

%%%%%%%%%%%%%%%%%%%%%%%% End Helper Commands %%%%%%%%%%%%%%%%%%%%%%%%%%%

%%%%%%%%%%%%%%%%%%%%%%%%% Begin CV Document %%%%%%%%%%%%%%%%%%%%%%%%%%%%

\begin{document}
\makeheading{Amirhosein Azarbakht}

%\begin{flushright}
%\textbf{(347) 276-0790 \\
%\textbf{\email{azarbaam@eecs.oregonstate.edu}\\}
%}\href{http://www.azarbakht.info/} {www.azarbakht.info}
%\end{flushright}

\section{Contact Information}

% NOTE: Mind where the & separators and \\ breaks are in the following
%       table. Table is one row made up of three parboxes. The left
%       parbox has address info, the middle parbox has a vertical bar,
%       and the right parbox has phone and electronic contact
%       information.
%
% MACROS: \rcollength is the width of the right column of the table
%             (adjust it to your liking; default is 1.85in).
%         \spacewidth is width of area between left and right boxes.
%         \spacechar is character used to produce perforated vertical
%             boundary between boxes.
%
\newlength{\rcollength}\setlength{\rcollength}{2.5in}%
\newlength{\spacewidth}\setlength{\spacewidth}{20pt}
\newcommand\spacechar{$|$}
%
\begin{tabular}[t]{@{}p{\textwidth-\rcollength-\spacewidth}@{}p{\spacewidth}@{}p{\rcollength}}%

% Address box
%\parbox{\textwidth-\rcollength-\spacewidth}{%
%1148 Kelley Engineering Center\\	
%\href{http://www.oregonstate.edu/}{Oregon State University}\\
%Corvallis, OR  97331 USA}
\parbox{\textwidth-\rcollength-\spacewidth}{%
Tel. (347) 276-0790\\
Tel. (When in Canada) 604-505-3993\\
\email{azarbaam@eecs.oregonstate.edu}
\href{http://eecs.oregonstate.edu/people/azarbakht} {http://eecs.oregonstate.edu/people/azarbakht}
}

% Cheesy perforated vertical bar between boxes
% Shorten by removing \spacechar's
%& \parbox{\spacewidth}{\spacechar\\\spacechar\\\spacechar\\\spacechar\\\spacechar} &

% Non-snail-mail contact information
& \parbox{\rcollength+\spacewidth}{%
\begin{flushright}
3048 Kelley Engineering Center\\
Corvallis, OR 97330\\
USA\\
\end{flushright}
}
\end{tabular}

%\section{Nationality}
%Canadian permanent resident, Iranian citizen

%\section{Specialties}
%\hrule
%\begin{innerlist}
%\item Social Network Analysis, Machine Learning
%\item User Experience (UX)
%\end{innerlist}
%\hrule

\section{Education}
		\textbf{Ph.D., Computer Science}     	\hfill \textit{(2011-present)\\}
		\href{http://www.oregonstate.edu/}{\textit{Oregon State University}}, Corvallis, OR USA\\
Thesis Title: Temporal Analysis and Visualization of Dynamic Collaboration Graphs of Open Source Software Development Community Forking | Advisor: Prof. Carlos Jensen\\
		\textbf{M.S., Computer Science}     	\hfill \textit{(2009-2011)\\}
		\href{http://www.chalmers.se/en/}{\textit{Chalmers University of Technology}}, Gothenburg Sweden\\
M.S. Thesis: An Evolutionary Algorithm for Computer-Generated Music Ranking\\
% | M.S. Advisors: Prof. Palle Dahlstedt \& Prof. Claes Strannegard\\
    	\textbf{B.S., Computer Engineering}     	\hfill \textit{(2004-2008)\\}
		\textit{Azad University of Central Tehran}, Tehran Iran\\
\hrule


\section{Area of Research}
My research focuses on analyzing software development communities. Particularly free/libre and open source software development communities. I am currently working on a project, under supervision of Prof. Carlos Jensen, that focuses on analyzing collaboration of software developers, especially the software development communities that have forked; forking is when a software community splits into two software project communities. Our goal is to identify unhealthy dynamics that hinder collaboration.\\
\hrule


\section{Professional Skills}
\textbf{Programming:} Java (expert), Python (proficient), C (proficient), \textsc{MATLAB} (expert), C++ (prior experience), Bash (proficient)\\
\textbf{Databases:} SQL, Hive\\
\textbf{Tools:} Git, Hadoop, \LaTeX{}\\
\textbf{Statistical Analysis:} R (expert)\\
\textbf{Platforms:} Linux \\
\hrule

\section{Research Experience}
\textbf{Software Engineering \& HCI Lab} \hfill \textbf{Research Assistant}\\
3048 EECS Department, Oregon State University \hfill \textit{(2012-present)}\\
\textit{Research on social dynamics of open source software development}\\

\textbf{Computer Vision Lab} \hfill \textbf{Research Assistant}\\
2126 EECS Department, Oregon State University \hfill \textit{(2011-2012)}\\
\textit{Research on activity recognition in videos}\\
\hrule

\section{Personal Projects}
\textbf{A Machine Learning Approach for Taming Compiler Fuzzers using Ensemble Clustering} \hfill \textit{(2014)}\newline
We developed a comparative approach to tame Compiler Fuzzers. The purpose of the project was to practice machine learning by doing, as well as to experience with different clustering techniques. We improved the state-of-the-art, as our approach found more unique bugs than the state-of-the-art.\\

\textbf{Augmented Reality Mirror: aMir} \hfill \textit{(2010)}\newline
We developed a prototype of a augmented mirror called “aMir”. The purpose of the project was to practice interaction design by doing, as well as to experience the value of prototyping. The project also brought together technical knowledge with more design-oriented thinking of IT.\\

\textbf{Corvallis Android App} \hfill \textit{(2013)}\newline
In context of the course “Mobile and Cloud Software Development” I developed an android app called “Corvallis” for the city of Corvallis. The purpose of the project was to practice mobile software development, as well as to create a means to keep track of the events in the little town I was living in.\\
\hrule

\section{Teaching Experience}
\textbf{User Experience (UX)} \hfill \textbf{Instructor}\\
Electrical Engineering \& Computer Science Department \hfill \textit{Summer 2014, Fall 2014, Winter 2015}\\
Oregon State University \hfill \textit{Spring 2015, Summer 2015}\\
% accomplished X by implementing Y which led to Z
%{\small I helped 320 post-baccalaureate students learn User Interface and UX design, to switch into CS careers. \\I maintained lectures, designed student engagement strategies, and supervised over 8 graduate teaching assistants. My teaching was ranked as exceptional (5.4/6), higher than the department average.}

\textbf{Data Structures} \hfill \textbf{Teaching Assistant}\\
Electrical Engineering \& Computer Science Department \hfill \textit{Fall 2012, Winter 2012, Fall 2013,}\\
Oregon State University \hfill \textit{Spring 2013, Spring 2014}\\
\hrule
% accomplished X by implementing Y which led to Z
%{\small I helped CS major sophomores learn C programming, by teaching recitations and helping them debug C code. I graded 2300 C programs, wrote shell scripts to automate compilation and runtime, and provided individual feedback to students on how to debug and fix C code.}

\section{Publications}
\begin{innerlist}
\item Azarbakht, A. and C. Jensen, ``Drawing the Big Picture: Temporal Visualization of Dynamic Collaboration Graphs of OSS Software Forks,'' \textit{Proc. 10th Int'l. Conf. Open Source Systems}, 2014.

\item Azarbakht, A. and C. Jensen, ``Temporal Visualization of Dynamic Collaboration Graphs of OSS Software Forks,'' \textit{Proc. Int'l. Network for Social Network Analysis Sunbelt conf.}, 2014.

\item Azarbakht, A., ``Drawing the Big Picture: Analyzing FLOSS Collaboration with Temporal Social Network Analysis,'' \textit{Proc. 9th Int'l. Symp. Open Collaboration}, 2013.

\item Azarbakht, A. and C. Jensen, ``Analyzing FOSS Collaboration \& Social Dynamics with Temporal Social Networks,'' \textit{Proc. 9th Int'l. Conf. Open Source Systems Doct. Cons.}, 2013.

\item Davidson, J, R. Naik, A. Mannan, A. Azarbakht, C. Jensen, ``Investigating Older Adults' Experiences with Contributing to Free/Open Source Software,'' \textit{Proc. IEEE Symp. Visual Languages and Human-Centric Computing}, 2014.

\item Azarbakht, A., ``Temporal Visualization of Collaborative Software Development in FOSS Forks,'' \textit{Proc. IEEE Symp. Visual Languages and Human-Centric Computing}, 2014.
\end{innerlist}

\section{Graduate Coursework}
\hrule
\begin{multicols}{2}
\begin{innerlist}
\item Machine Learning
\item Artificial Intelligence
\item Stochastic Optimization
\item Statistical Methods of Data Analysis
\item Theory of Statistics I \& II
\item Computer Vision
%\item HCI meets Software Development: The Case Study (CS569)
\item Algorithms \& Data Structures
%\item Open Source Software Development (CS519)
\item Mobile \& Cloud Software Development
\item Unix Internals: FreeBSD Operating System
%\item Ubiquitous Computing 
\item Qualitative \& Quantitative Research Methods
\end{innerlist}
\end{multicols}
\hrule


%\section{References} 
%Available upon request

%\section{Volunteer Service}
%Department Steward for the American Federation of Teachers (AFT-Local 6069) \hfill \textit{(2012, 2013, 2014)}\\


%\section{References}
%Available upon request

%
%\section{\textsc{Activities}}
%\begin{innerlist}
%\item Tennis
%\item Salsa dancing
%\end{innerlist}

%\section{Academic Appointment}
%		\textbf{Research Assistantship}\\
%		\textit{Oregon State University, Computer vision lab}
%    	\hfill \textit{September~2011 to present}\\
%    	Tracking interest points in images and videos, Video segmentation, Classification, Clustering, Support vector machines, Pedestrian detection, Activity recognition in videos, 
%	
% The ``More Info'' section may not be necessary; make sure it's short
% so it doesn't prevent people from seeing references available to
% contact.

%\section{Languages}
%\begin{innerlist}
%\item \textsc{English}
%\item \textsc{Persian}
%\item \textsc{French} (Intermediate)
%\item \textsc{Swedish} (Elementary)
%
%\end{innerlist}

%\section{Certifications}
%\begin{innerlist}
%\item IELTS
%\hfill \textit{Gothenburg, Sweden, September~2009}\\
%\textit{Scores: Listening 8, Speaking 7.5, Reading 7.5, Writing 7, Overall 7.5}
%
%\item GRE
%    	\hfill \textit{Stockholm, Sweden, October~2009}\\
%\textit{Verbal reasoning  390, Quantitative reasoning  790, Analytical Writing  3.5}
%
%\item T\'{E}F (Test d'\'{E}valuation de Fran{\c c}ais)
%    	\hfill \textit{Stockholm, Sweden, May~2011}\\
%\textsc{from Chambre de commerce et d'industrie de Paris}
%
%\textit{Compréhension orale 141, Compréhension écrite 97, Lexique et Structure 72}
%
%\item A1 level certificate in Swedish language
%    	\hfill \textit{Gothenburg, Sweden, June~2011}\\
%\textsc{from University of Gothenburg}
%\end{innerlist}



%\textbf{Science and Research Methodology}
%Qualitative \& Quantitative research methods: Conducting Interview, Survey, Case Study, Experiment. Usability evaluation \& User testing, and scientific paper writing. I conducted two in-depth interviews; one semi-structured interview; a 16-hour observational study; four user testings; one heuristic evaluation; transcribed 12 pages of interview records; and wrote three ACM standard papers.\\
%
%Artificial Intelligence I
%Classical AI theory \& methods, Intelligent Agents, Heuristic Search
%
%Artificial Intelligence II (biologically-inspired stochastic optimization methods)
%
%Evolutionary Algorithms (EAs); Ant Colony Optimization (ACO); Particle Swarm Optimization (PSO); Artificial Neural Networks (ANNs).
%    I implemented a Genetic Algorithm(GA), a PSO, and an ACO to solve the NP-hard Traveling Salesman Problem (TSP) in MATLAB
%
%Physical Computing at Interaction Design Collegium
%Designing \& building prototypes of computational nature, using Arduino, sensors, actuators and smart material, e.g. wearable fabrics. Projects: iFlute,  Compose:Me,  Power-Sleeve for Tetris
%
%Ubiquitous Computing at Interaction Design Collegium
%Designing and developing a working prototype of an interactive embedded computer system using novel interface components -- within a budget of 1500 SEK. We developed an interactive augmented mirror, called aMir
%
%Unix Internals
%The design and implementation of the FreeBSD operating system core: kernel; processes; virtual memory; I/O system; local filesystems; devices; NFS; terminal handling; sockets; network communication; network protocols; startup
%
%Theory of Statistics I (ST561)
%Notions of probability; expectations of functions of random variables; expectation and variance; moments and moment generating functions; multivariate random variables; joint, marginal and conditional distributions; sampling distributions; convolution; stochastic convergence; order statistics
%
%Theory of Statistics II (ST562)
%Sufficiency; completeness; ancillary; exponential families; location and scale families; point estimation: maximum likelihood, Bayes and unbiased estimators
%
%Numerical Linear Algebra (MTH551)
%Direct and iterative methods for solving linear systems; least square solution of overdetermined systems; eigenvalues and eigenvectors
%
%Algorithms \& Data Structures (CS515)
%Divide and conquer; greedy algorithms; matroids; dynamic programming; randomized algorithms; randomized min-cut; network flow; linear programming
%
%Computer Vision (CS556)
%Image features and descriptors; feature extraction; Hungarian algorithm; color; edges; shape descriptors; shape matching; dynamic time warping; perceptual grouping and Gestalt laws; image segmentation; normalized cut; mean-shift; matching: typical formulations; matching as convex optimization; imaging process; geometric primitives; 2D and 3D transformations; projections; epipolar geometry; calibration; 2D homography; object detection; bag-of-words; tracking; clustering
%
%Open Source Software Development (CS519)
%History of Free and Open Source Software; licensing and intellectual property; ethics and etiquette; challenges and limitations; real hands-on experience
%
%Mobile \& Cloud Software Development (CS569)
%iPhone and Android app development with focus on performance; usability; reliability; security; efficient storage of replicated data on the cloud; reliable synchronization of offline data; mobile-optimized user interfaces (UI); I developed a mobile app for Android OS called Corvallis, available on the Google Android market (here)
%\begin{innerlist}
%\item Qualitative \& Quantitative research methods
%\item Open Source Software Development (CS519)
%\item Artificial Intelligence I
%\item Artificial Intelligence II (Stochastic Optimization)
%\item Ubiquitous Computing
%\item Physical Computing
%\item Unix Internals
%\item Theory of Statistics I (ST561)
%\item Theory of Statistics II (ST562)
%\item Numerical Linear Algebra (MTH551)
%\item Algorithms \& Data Structures (CS515)
%\item Computer Vision (CS556)
%\item Mobile \& Cloud Software Development (CS569)
%\end{innerlist}


%
%\textbf{Stochastic optimization}
%Evolutionary Algorithms (EAs); Ant Colony Optimization (ACO); Particle Swarm Optimization (PSO); Artificial Neural Networks (ANNs).
%\textit{I implemented a Genetic Algorithm(GA), a Particle Swarm Optimization (PSO), and an (Ant Colony Optimization)ACO to solve the NP-hard Traveling Salesman Problem (TSP) in MATLAB.}\\
%
%\textbf{Computer Vision}
%Extracting image features (Harris affine corners, Hessian regions, SURF \& MSER detectors); SIFT descriptors; Matching using Hungarian algorithm; Shape descriptors (shape-context \& beam-angle); Image segmentation using normalized cuts, Video Segmentation using normalized cut \& Nystrom method.\\
%
%\textbf{HCI: Physical Computing}
%Designing \& building prototypes of computational nature, using Arduino, sensors, actuators and smart material, e. g. wearable fabrics. Projects: \href{http://chalmersphyscomp09.wordpress.com/2009/09/20/iflute/}{iFlute}, \href{http://chalmersphyscomp09.wordpress.com/2009/10/17/composeme/}{Compose:Me}, \href{http://chalmersphyscomp09.wordpress.com/2009/10/13/the-power-sleeve-for-tetris/}{Power-Sleeve for Tetris}.\\
%
%\textbf{HCI: Ubiquitous Computing}
%Designing and developing a working prototype of an interactive embedded computer system using novel interface components -- within a budget of 1500 SEK.
%We developed an interactive augmented mirror, called \href{http://www.azarbakht.info/augmentedmirror/}{aMir}.\\ 
%See here: \url{http://www.azarbakht.info/augmentedmirror/}.\\

%\section{Scholarships}
%	The Adlerbert Scholarship Foundation
%	\hfill \textit{Gothenburg, Sweden, 2010}

%\section{Volunteer Work}
%\begin{innerlist}
%\item Orientation leader for 20+ exchange \& international students.
%\hfill \textit{Chalmers University's International Reception Committee. Sweden, 2011}
%\item Three-times Elected Steward for the Coalition of Graduate Employees (CGE).
%\hfill \textit{2012, 2013, 2014}
%
%			        
%\end{innerlist}

%\section{Memberships}
%\begin{innerlist}
%\item \textsc{Association for Computing Machinery (ACM)}
%\item \textsc{American Federation of Teachers (AFT)}
%\item Rumbanana Cuban Casino Salsa Team (I dance salsa)
%\item OSU Tennis Club
%
%\end{innerlist}

%\section{Contact Information}
%
%% NOTE: Mind where the & separators and \\ breaks are in the following
%%       table. Table is one row made up of three parboxes. The left
%%       parbox has address info, the middle parbox has a vertical bar,
%%       and the right parbox has phone and electronic contact
%%       information.
%%
%% MACROS: \rcollength is the width of the right column of the table
%%             (adjust it to your liking; default is 1.85in).
%%         \spacewidth is width of area between left and right boxes.
%%         \spacechar is character used to produce perforated vertical
%%             boundary between boxes.
%%
%\newlength{\rcollength}\setlength{\rcollength}{2.5in}%
%\newlength{\spacewidth}\setlength{\spacewidth}{20pt}
%\newcommand\spacechar{$|$}
%%
%\begin{tabular}[t]{@{}p{\textwidth-\rcollength-\spacewidth}@{}p{\spacewidth}@{}p{\rcollength}}%
%
%% Address box
%%\parbox{\textwidth-\rcollength-\spacewidth}{%
%%1148 Kelley Engineering Center\\	
%%\href{http://www.oregonstate.edu/}{Oregon State University}\\
%%Corvallis, OR  97331 USA}
%\parbox{\textwidth-\rcollength-\spacewidth}{%
%1148 Kelley Engineering Center\\	
%Corvallis, OR  97331\\
%%U.S.A.\\
%}
%
%% Cheesy perforated vertical bar between boxes
%% Shorten by removing \spacechar's
%%& \parbox{\spacewidth}{\spacechar\\\spacechar\\\spacechar\\\spacechar\\\spacechar} &
%
%% Non-snail-mail contact information
%& \parbox{\rcollength}{%
%\textit{Mobile:} (347) 276-0790 \\
%\textit{E-mail:} \email{azarbaam@eecs.oregonstate.edu}\\
%\textit{Webpage:} 
%\href{http://eecs.oregonstate.edu/~azarbaam/} {http://eecs.oregonstate.edu/\textasciitilde azarbaam}
%%&
%%\textit{E-mail:} \email{amir@azarbakht.info}\\
%%\textit{WWW:} \href{http://www.azarbakht.info/} {http://www.azarbakht.info}
%%\href{http://research.engr.oregonstate.edu/hci/content/amir-azarbakht}{http://research.engr.oregonstate.edu/hci/content/amir-azarbakht}
%} 
%
%\textit{LinkedIn:}  
%\href{http://www.linkedin.com/profile/view?id=63719178}{http://www.linkedin.com/profile/view?id=63719178}
%
%%http://research.engr.oregonstate.edu/hci/content/amir-azarbakht
%
%\end{tabular}

%%
%% In modern CV's, it seems like ``Objective'' is frowned upon. Instead,
%% incorporate it into a well-constructed cover letter. The ``More
%% information'' can go at the end of the CV, but it should not distract
%% from the section giving references available to contact.
%%
%
% \section{Objective}
%
% Placement in an academic position (i.e., faculty, postdoctoral, or
% research scientist) that allows for advanced research in distributed
% complex adaptive systems (i.e., modeling, analysis, design, and
% verification) with a particular focus on the control of engineered
% agents (e.g., for communications, control, software, electronics, and
% sustainability) and the analysis of biological phenomena (e.g.,
% self-organization, ecological rationality)
% \begin{innerlist}
% \item More information and auxiliary documents can be found at\\\url{http://www.tedpavlic.com/facjobsearch/}
% \end{innerlist}
%

%\section{Research Interests}
%Computer vision, Human-computer interaction, Object detection and recognition, Video analysis, activity recognition, Interaction design, Augmented reality, Tangible interfaces, Intelligent systems
%
%\section{Academic Appointments}
%
%\textbf{Postdoctoral Researcher} \hfill {September 2010 to present}
%\begin{innerlist}
%
%    \item[] \href{http://www.cse.ohio-state.edu/}{Department of Computer Science and Engineering},
%            \href{http://www.osu.edu/}{The Ohio State University}
%    \begin{innerlist}
%        \item \href{http://www.nfs.gov/}{National Science Foundation} Cyber-Physical Systems (ENG, \href{http://www.nsf.gov/div/index.jsp?div=eccs}{ECCS})
%        \begin{innerlist}
%            \item[$-$] ``Autonomous Driving in Mixed-Traffic Urban Environments''
%                (grant~\href{http://www.nsf.gov/awardsearch/showAward.do?AwardNumber=0931669}{\#0931669})
%            \item[$-$] Supervisor (co-PI):
%                \href{http://www.cse.ohio-state.edu/~paolo/}%
%                     {Professor Paolo A.~G.~Sivilotti}
%            \item[$-$] PI:
%                \href{http://www.ece.ohio-state.edu/~umit/}%
%                     {Professor \"{U}mit \"{O}zg\"{u}ner}
%        \end{innerlist}
%    \end{innerlist}
%
%\end{innerlist}

%
%\section{Areas of Expertise}
%\begin{innerlist}
%\item Social Network Analysis (SNA)
%\item Human-Computer Interaction (HCI)
%
%\end{innerlist}

%
%\section{Further Info}
%Available on my webpage: \href{http://www.azarbakht.info/} {www.azarbakht.info}

%\section{References}
%Reference are available upon request

\end{document}

%%%%%%%%%%%%%%%%%%%%%%%%%% End CV Document %%%%%%%%%%%%%%%%%%%%%%%%%%%%%

%----------------------------------------------------------------------%
% The following is copyright and licensing information for
% redistribution of this LaTeX source code; it also includes a liability
% statement. If this source code is not being redistributed to others,
% it may be omitted. It has no effect on the function of the above code.
%----------------------------------------------------------------------%
% Copyright (c) 2007, 2008, 2009, 2010, 2011 by Theodore P. Pavlic
%
% Unless otherwise expressly stated, this work is licensed under the
% Creative Commons Attribution-Noncommercial 3.0 United States License. To
% view a copy of this license, visit
% http://creativecommons.org/licenses/by-nc/3.0/us/ or send a letter to
% Creative Commons, 171 Second Street, Suite 300, San Francisco,
% California, 94105, USA.
%
% THE SOFTWARE IS PROVIDED "AS IS", WITHOUT WARRANTY OF ANY KIND, EXPRESS
% OR IMPLIED, INCLUDING BUT NOT LIMITED TO THE WARRANTIES OF
% MERCHANTABILITY, FITNESS FOR A PARTICULAR PURPOSE AND NONINFRINGEMENT.
% IN NO EVENT SHALL THE AUTHORS OR COPYRIGHT HOLDERS BE LIABLE FOR ANY
% CLAIM, DAMAGES OR OTHER LIABILITY, WHETHER IN AN ACTION OF CONTRACT,
% TORT OR OTHERWISE, ARISING FROM, OUT OF OR IN CONNECTION WITH THE
% SOFTWARE OR THE USE OR OTHER DEALINGS IN THE SOFTWARE.
%----------------------------------------------------------------------%
